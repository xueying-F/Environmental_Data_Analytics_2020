\documentclass[]{article}
\usepackage{lmodern}
\usepackage{amssymb,amsmath}
\usepackage{ifxetex,ifluatex}
\usepackage{fixltx2e} % provides \textsubscript
\ifnum 0\ifxetex 1\fi\ifluatex 1\fi=0 % if pdftex
  \usepackage[T1]{fontenc}
  \usepackage[utf8]{inputenc}
\else % if luatex or xelatex
  \ifxetex
    \usepackage{mathspec}
  \else
    \usepackage{fontspec}
  \fi
  \defaultfontfeatures{Ligatures=TeX,Scale=MatchLowercase}
\fi
% use upquote if available, for straight quotes in verbatim environments
\IfFileExists{upquote.sty}{\usepackage{upquote}}{}
% use microtype if available
\IfFileExists{microtype.sty}{%
\usepackage{microtype}
\UseMicrotypeSet[protrusion]{basicmath} % disable protrusion for tt fonts
}{}
\usepackage[margin=2.54cm]{geometry}
\usepackage{hyperref}
\hypersetup{unicode=true,
            pdftitle={Assignment 6: GLMs week 1 (t-test and ANOVA)},
            pdfauthor={Xueying Feng},
            pdfborder={0 0 0},
            breaklinks=true}
\urlstyle{same}  % don't use monospace font for urls
\usepackage{color}
\usepackage{fancyvrb}
\newcommand{\VerbBar}{|}
\newcommand{\VERB}{\Verb[commandchars=\\\{\}]}
\DefineVerbatimEnvironment{Highlighting}{Verbatim}{commandchars=\\\{\}}
% Add ',fontsize=\small' for more characters per line
\usepackage{framed}
\definecolor{shadecolor}{RGB}{248,248,248}
\newenvironment{Shaded}{\begin{snugshade}}{\end{snugshade}}
\newcommand{\AlertTok}[1]{\textcolor[rgb]{0.94,0.16,0.16}{#1}}
\newcommand{\AnnotationTok}[1]{\textcolor[rgb]{0.56,0.35,0.01}{\textbf{\textit{#1}}}}
\newcommand{\AttributeTok}[1]{\textcolor[rgb]{0.77,0.63,0.00}{#1}}
\newcommand{\BaseNTok}[1]{\textcolor[rgb]{0.00,0.00,0.81}{#1}}
\newcommand{\BuiltInTok}[1]{#1}
\newcommand{\CharTok}[1]{\textcolor[rgb]{0.31,0.60,0.02}{#1}}
\newcommand{\CommentTok}[1]{\textcolor[rgb]{0.56,0.35,0.01}{\textit{#1}}}
\newcommand{\CommentVarTok}[1]{\textcolor[rgb]{0.56,0.35,0.01}{\textbf{\textit{#1}}}}
\newcommand{\ConstantTok}[1]{\textcolor[rgb]{0.00,0.00,0.00}{#1}}
\newcommand{\ControlFlowTok}[1]{\textcolor[rgb]{0.13,0.29,0.53}{\textbf{#1}}}
\newcommand{\DataTypeTok}[1]{\textcolor[rgb]{0.13,0.29,0.53}{#1}}
\newcommand{\DecValTok}[1]{\textcolor[rgb]{0.00,0.00,0.81}{#1}}
\newcommand{\DocumentationTok}[1]{\textcolor[rgb]{0.56,0.35,0.01}{\textbf{\textit{#1}}}}
\newcommand{\ErrorTok}[1]{\textcolor[rgb]{0.64,0.00,0.00}{\textbf{#1}}}
\newcommand{\ExtensionTok}[1]{#1}
\newcommand{\FloatTok}[1]{\textcolor[rgb]{0.00,0.00,0.81}{#1}}
\newcommand{\FunctionTok}[1]{\textcolor[rgb]{0.00,0.00,0.00}{#1}}
\newcommand{\ImportTok}[1]{#1}
\newcommand{\InformationTok}[1]{\textcolor[rgb]{0.56,0.35,0.01}{\textbf{\textit{#1}}}}
\newcommand{\KeywordTok}[1]{\textcolor[rgb]{0.13,0.29,0.53}{\textbf{#1}}}
\newcommand{\NormalTok}[1]{#1}
\newcommand{\OperatorTok}[1]{\textcolor[rgb]{0.81,0.36,0.00}{\textbf{#1}}}
\newcommand{\OtherTok}[1]{\textcolor[rgb]{0.56,0.35,0.01}{#1}}
\newcommand{\PreprocessorTok}[1]{\textcolor[rgb]{0.56,0.35,0.01}{\textit{#1}}}
\newcommand{\RegionMarkerTok}[1]{#1}
\newcommand{\SpecialCharTok}[1]{\textcolor[rgb]{0.00,0.00,0.00}{#1}}
\newcommand{\SpecialStringTok}[1]{\textcolor[rgb]{0.31,0.60,0.02}{#1}}
\newcommand{\StringTok}[1]{\textcolor[rgb]{0.31,0.60,0.02}{#1}}
\newcommand{\VariableTok}[1]{\textcolor[rgb]{0.00,0.00,0.00}{#1}}
\newcommand{\VerbatimStringTok}[1]{\textcolor[rgb]{0.31,0.60,0.02}{#1}}
\newcommand{\WarningTok}[1]{\textcolor[rgb]{0.56,0.35,0.01}{\textbf{\textit{#1}}}}
\usepackage{graphicx,grffile}
\makeatletter
\def\maxwidth{\ifdim\Gin@nat@width>\linewidth\linewidth\else\Gin@nat@width\fi}
\def\maxheight{\ifdim\Gin@nat@height>\textheight\textheight\else\Gin@nat@height\fi}
\makeatother
% Scale images if necessary, so that they will not overflow the page
% margins by default, and it is still possible to overwrite the defaults
% using explicit options in \includegraphics[width, height, ...]{}
\setkeys{Gin}{width=\maxwidth,height=\maxheight,keepaspectratio}
\IfFileExists{parskip.sty}{%
\usepackage{parskip}
}{% else
\setlength{\parindent}{0pt}
\setlength{\parskip}{6pt plus 2pt minus 1pt}
}
\setlength{\emergencystretch}{3em}  % prevent overfull lines
\providecommand{\tightlist}{%
  \setlength{\itemsep}{0pt}\setlength{\parskip}{0pt}}
\setcounter{secnumdepth}{0}
% Redefines (sub)paragraphs to behave more like sections
\ifx\paragraph\undefined\else
\let\oldparagraph\paragraph
\renewcommand{\paragraph}[1]{\oldparagraph{#1}\mbox{}}
\fi
\ifx\subparagraph\undefined\else
\let\oldsubparagraph\subparagraph
\renewcommand{\subparagraph}[1]{\oldsubparagraph{#1}\mbox{}}
\fi

%%% Use protect on footnotes to avoid problems with footnotes in titles
\let\rmarkdownfootnote\footnote%
\def\footnote{\protect\rmarkdownfootnote}

%%% Change title format to be more compact
\usepackage{titling}

% Create subtitle command for use in maketitle
\providecommand{\subtitle}[1]{
  \posttitle{
    \begin{center}\large#1\end{center}
    }
}

\setlength{\droptitle}{-2em}

  \title{Assignment 6: GLMs week 1 (t-test and ANOVA)}
    \pretitle{\vspace{\droptitle}\centering\huge}
  \posttitle{\par}
    \author{Xueying Feng}
    \preauthor{\centering\large\emph}
  \postauthor{\par}
    \date{}
    \predate{}\postdate{}
  

\begin{document}
\maketitle

\hypertarget{overview}{%
\subsection{OVERVIEW}\label{overview}}

This exercise accompanies the lessons in Environmental Data Analytics on
t-tests and ANOVAs.

\hypertarget{directions}{%
\subsection{Directions}\label{directions}}

\begin{enumerate}
\def\labelenumi{\arabic{enumi}.}
\tightlist
\item
  Change ``Student Name'' on line 3 (above) with your name.
\item
  Work through the steps, \textbf{creating code and output} that fulfill
  each instruction.
\item
  Be sure to \textbf{answer the questions} in this assignment document.
\item
  When you have completed the assignment, \textbf{Knit} the text and
  code into a single PDF file.
\item
  After Knitting, submit the completed exercise (PDF file) to the
  dropbox in Sakai. Add your last name into the file name (e.g.,
  ``Salk\_A06\_GLMs\_Week1.Rmd'') prior to submission.
\end{enumerate}

The completed exercise is due on Tuesday, February 18 at 1:00 pm.

\hypertarget{set-up-your-session}{%
\subsection{Set up your session}\label{set-up-your-session}}

\begin{enumerate}
\def\labelenumi{\arabic{enumi}.}
\item
  Check your working directory, load the \texttt{tidyverse},
  \texttt{cowplot}, and \texttt{agricolae} packages, and import the
  NTL-LTER\_Lake\_Nutrients\_PeterPaul\_Processed.csv dataset.
\item
  Change the date column to a date format. Call up \texttt{head} of this
  column to verify.
\end{enumerate}

\begin{Shaded}
\begin{Highlighting}[]
\CommentTok{#1}
\KeywordTok{getwd}\NormalTok{()}
\end{Highlighting}
\end{Shaded}

\begin{verbatim}
## [1] "/Users/ethel/Desktop/Environ 872/Environmental_Data_Analytics_2020"
\end{verbatim}

\begin{Shaded}
\begin{Highlighting}[]
\CommentTok{#install.packages("agricolae")}
\CommentTok{#install.packages("tidyverse")}
\CommentTok{#install.packages("cowplot")}

\KeywordTok{library}\NormalTok{(tidyverse)}
\KeywordTok{library}\NormalTok{(cowplot)}
\KeywordTok{library}\NormalTok{(agricolae)}

\NormalTok{PeterPaul.nutrient <-}
\StringTok{  }\KeywordTok{read.csv}\NormalTok{(}\StringTok{"./Data/Processed/NTL-LTER_Lake_Nutrients_PeterPaul_Processed.csv"}\NormalTok{)}

\CommentTok{#2}
\NormalTok{PeterPaul.nutrient}\OperatorTok{$}\NormalTok{sampledate <-}\StringTok{ }\KeywordTok{as.Date}\NormalTok{(PeterPaul.nutrient}\OperatorTok{$}\NormalTok{sampledate, }\DataTypeTok{format =} \StringTok{"%Y-%m-%d"}\NormalTok{)}
\KeywordTok{class}\NormalTok{(PeterPaul.nutrient}\OperatorTok{$}\NormalTok{sampledate)}
\end{Highlighting}
\end{Shaded}

\begin{verbatim}
## [1] "Date"
\end{verbatim}

\hypertarget{wrangle-your-data}{%
\subsection{Wrangle your data}\label{wrangle-your-data}}

\begin{enumerate}
\def\labelenumi{\arabic{enumi}.}
\setcounter{enumi}{2}
\tightlist
\item
  Wrangle your dataset so that it contains only surface depths and only
  the years 1993-1996, inclusive. Set month as a factor.
\end{enumerate}

\begin{Shaded}
\begin{Highlighting}[]
\NormalTok{PeterPaul.surface93_}\DecValTok{96}\NormalTok{ <-}\StringTok{ }\KeywordTok{filter}\NormalTok{(PeterPaul.nutrient, depth }\OperatorTok{==}\StringTok{ "0"} 
                                 \OperatorTok{&}\StringTok{ }\NormalTok{year4 }\OperatorTok\StringTok{ }\KeywordTok{c}\NormalTok{(}\StringTok{"1993"}\NormalTok{,}\StringTok{"1994"}\NormalTok{,}\StringTok{"1995"}\NormalTok{,}\StringTok{"1996"}\NormalTok{) ) }


\KeywordTok{class}\NormalTok{(PeterPaul.surface93_}\DecValTok{96}\OperatorTok{$}\NormalTok{month)}
\end{Highlighting}
\end{Shaded}

\begin{verbatim}
## [1] "integer"
\end{verbatim}

\begin{Shaded}
\begin{Highlighting}[]
\NormalTok{PeterPaul.surface93_}\DecValTok{96}\OperatorTok{$}\NormalTok{month <-}\StringTok{ }\KeywordTok{as.factor}\NormalTok{(PeterPaul.surface93_}\DecValTok{96}\OperatorTok{$}\NormalTok{month)}
\KeywordTok{class}\NormalTok{(PeterPaul.surface93_}\DecValTok{96}\OperatorTok{$}\NormalTok{month)}
\end{Highlighting}
\end{Shaded}

\begin{verbatim}
## [1] "factor"
\end{verbatim}

\hypertarget{analysis}{%
\subsection{Analysis}\label{analysis}}

Peter Lake was manipulated with additions of nitrogen and phosphorus
over the years 1993-1996 in an effort to assess the impacts of
eutrophication in lakes. You are tasked with finding out if nutrients
are significantly higher in Peter Lake than Paul Lake, and if these
potential differences in nutrients vary seasonally (use month as a
factor to represent seasonality). Run two separate tests for TN and TP.

\begin{enumerate}
\def\labelenumi{\arabic{enumi}.}
\setcounter{enumi}{3}
\tightlist
\item
  Which application of the GLM will you use (t-test, one-way ANOVA,
  two-way ANOVA with main effects, or two-way ANOVA with interaction
  effects)? Justify your choice.
\end{enumerate}

\begin{quote}
Answer: I will use two way ANOVA. A two-way ANOVA compares multiple
groups of two factors.
\end{quote}

\begin{enumerate}
\def\labelenumi{\arabic{enumi}.}
\setcounter{enumi}{4}
\item
  Run your test for TN. Include examination of groupings and consider
  interaction effects, if relevant.
\item
  Run your test for TP. Include examination of groupings and consider
  interaction effects, if relevant.
\end{enumerate}

\begin{Shaded}
\begin{Highlighting}[]
\CommentTok{#5}
\NormalTok{nutrient_seasonally <-}\StringTok{ }\NormalTok{PeterPaul.surface93_}\DecValTok{96} \OperatorTok
\StringTok{  }\KeywordTok{select}\NormalTok{(lakename, month, year4, tn_ug, tp_ug)}


\CommentTok{# Interaction effects}
\NormalTok{TN.anova}\FloatTok{.2}\NormalTok{way <-}\StringTok{ }\KeywordTok{lm}\NormalTok{(}\DataTypeTok{data =}\NormalTok{ nutrient_seasonally, tn_ug }\OperatorTok{~}\StringTok{ }\NormalTok{lakename }\OperatorTok{*}\StringTok{ }\NormalTok{month)}
\KeywordTok{summary}\NormalTok{(TN.anova}\FloatTok{.2}\NormalTok{way)}
\end{Highlighting}
\end{Shaded}

\begin{verbatim}
## 
## Call:
## lm(formula = tn_ug ~ lakename * month, data = nutrient_seasonally)
## 
## Residuals:
##     Min      1Q  Median      3Q     Max 
## -357.88 -118.10  -10.41   50.58 1353.86 
## 
## Coefficients:
##                           Estimate Std. Error t value Pr(>|t|)   
## (Intercept)                 300.51     106.30   2.827   0.0057 **
## lakenamePeter Lake           84.43     144.86   0.583   0.5614   
## month6                       23.61     123.64   0.191   0.8489   
## month7                       53.12     127.05   0.418   0.6768   
## month8                       36.00     127.05   0.283   0.7775   
## month9                      105.82     184.11   0.575   0.5668   
## lakenamePeter Lake:month6   200.49     170.90   1.173   0.2436   
## lakenamePeter Lake:month7   271.82     176.18   1.543   0.1261   
## lakenamePeter Lake:month8   325.05     174.20   1.866   0.0651 . 
## lakenamePeter Lake:month9    59.70     278.35   0.214   0.8306   
## ---
## Signif. codes:  0 '***' 0.001 '**' 0.01 '*' 0.05 '.' 0.1 ' ' 1
## 
## Residual standard error: 260.4 on 97 degrees of freedom
##   (23 observations deleted due to missingness)
## Multiple R-squared:  0.3285, Adjusted R-squared:  0.2662 
## F-statistic: 5.272 on 9 and 97 DF,  p-value: 7.729e-06
\end{verbatim}

\begin{Shaded}
\begin{Highlighting}[]
\CommentTok{#6}
\CommentTok{# Interaction effects}
\NormalTok{TP.anova}\FloatTok{.2}\NormalTok{way <-}\StringTok{ }\KeywordTok{aov}\NormalTok{(}\DataTypeTok{data =}\NormalTok{ nutrient_seasonally, tp_ug }\OperatorTok{~}\StringTok{ }\NormalTok{lakename }\OperatorTok{*}\StringTok{ }\NormalTok{month)}
\KeywordTok{summary}\NormalTok{(TP.anova}\FloatTok{.2}\NormalTok{way)}
\end{Highlighting}
\end{Shaded}

\begin{verbatim}
##                 Df Sum Sq Mean Sq F value Pr(>F)    
## lakename         1  10228   10228  98.914 <2e-16 ***
## month            4    813     203   1.965 0.1043    
## lakename:month   4   1014     254   2.452 0.0496 *  
## Residuals      119  12305     103                   
## ---
## Signif. codes:  0 '***' 0.001 '**' 0.01 '*' 0.05 '.' 0.1 ' ' 1
## 1 observation deleted due to missingness
\end{verbatim}

\begin{Shaded}
\begin{Highlighting}[]
\CommentTok{# Run a post-hoc test for pairwise differences}
\KeywordTok{TukeyHSD}\NormalTok{(TP.anova}\FloatTok{.2}\NormalTok{way)}
\end{Highlighting}
\end{Shaded}

\begin{verbatim}
##   Tukey multiple comparisons of means
##     95% family-wise confidence level
## 
## Fit: aov(formula = tp_ug ~ lakename * month, data = nutrient_seasonally)
## 
## $lakename
##                          diff      lwr      upr p adj
## Peter Lake-Paul Lake 17.80939 14.26365 21.35513     0
## 
## $month
##           diff         lwr       upr     p adj
## 6-5  6.3451786  -2.8038335 15.494191 0.3119085
## 7-5  8.8661326  -0.2828796 18.015145 0.0622967
## 8-5  4.8191843  -4.2626118 13.900980 0.5839528
## 9-5  5.4951391  -6.7194172 17.709695 0.7243206
## 7-6  2.5209540  -4.2125367  9.254445 0.8376355
## 8-6 -1.5259943  -8.1678685  5.115880 0.9688094
## 9-6 -0.8500395 -11.3776631  9.677584 0.9994372
## 8-7 -4.0469483 -10.6888225  2.594926 0.4453729
## 9-7 -3.3709935 -13.8986170  7.156630 0.9012092
## 9-8  0.6759548  -9.7933076 11.145217 0.9997679
## 
## $`lakename:month`
##                                  diff         lwr         upr     p adj
## Peter Lake:5-Paul Lake:5    4.3135714 -13.9293175  22.5564604 0.9989515
## Paul Lake:6-Paul Lake:5    -0.9178824 -16.4886641  14.6528993 1.0000000
## Peter Lake:6-Paul Lake:5   16.8838889   1.4263507  32.3414270 0.0206973
## Paul Lake:7-Paul Lake:5    -1.7271111 -17.1846493  13.7304270 0.9999981
## Peter Lake:7-Paul Lake:5   22.9304706   7.3596889  38.5012523 0.0002415
## Paul Lake:8-Paul Lake:5    -2.0872222 -17.5447604  13.3703159 0.9999902
## Peter Lake:8-Paul Lake:5   15.0200000  -0.3355071  30.3755071 0.0607728
## Paul Lake:9-Paul Lake:5    -0.7380000 -20.5935673  19.1175673 1.0000000
## Peter Lake:9-Paul Lake:5   14.7452500  -6.4208558  35.9113558 0.4316694
## Paul Lake:6-Peter Lake:5   -5.2314538 -19.9572479   9.4943403 0.9787107
## Peter Lake:6-Peter Lake:5  12.5703175  -2.0356832  27.1763181 0.1571717
## Paul Lake:7-Peter Lake:5   -6.0406825 -20.6466832   8.5653181 0.9437275
## Peter Lake:7-Peter Lake:5  18.6168992   3.8911050  33.3426933 0.0032014
## Paul Lake:8-Peter Lake:5   -6.4007937 -21.0067943   8.2052070 0.9208652
## Peter Lake:8-Peter Lake:5  10.7064286  -3.7915495  25.2044066 0.3464892
## Paul Lake:9-Peter Lake:5   -5.0515714 -24.2516579  14.1485150 0.9975850
## Peter Lake:9-Peter Lake:5  10.4316786 -10.1207861  30.9841433 0.8273658
## Peter Lake:6-Paul Lake:6   17.8017712   6.7120688  28.8914737 0.0000401
## Paul Lake:7-Paul Lake:6    -0.8092288 -11.8989312  10.2804737 1.0000000
## Peter Lake:7-Paul Lake:6   23.8483529  12.6013419  35.0953640 0.0000000
## Paul Lake:8-Paul Lake:6    -1.1693399 -12.2590423   9.9203626 0.9999989
## Peter Lake:8-Paul Lake:6   15.9378824   4.9908457  26.8849190 0.0003006
## Paul Lake:9-Paul Lake:6     0.1798824 -16.5021309  16.8618956 1.0000000
## Peter Lake:9-Paul Lake:6   15.6631324  -2.5591082  33.8853729 0.1584032
## Paul Lake:7-Peter Lake:6  -18.6110000 -29.5411300  -7.6808700 0.0000101
## Peter Lake:7-Peter Lake:6   6.0465817  -5.0431207  17.1362841 0.7595330
## Paul Lake:8-Peter Lake:6  -18.9711111 -29.9012412  -8.0409811 0.0000062
## Peter Lake:8-Peter Lake:6  -1.8638889 -12.6492426   8.9214648 0.9999197
## Paul Lake:9-Peter Lake:6  -17.6218889 -34.1982518  -1.0455259 0.0276305
## Peter Lake:9-Peter Lake:6  -2.1386389 -20.2642090  15.9869312 0.9999970
## Peter Lake:7-Paul Lake:7   24.6575817  13.5678793  35.7472841 0.0000000
## Paul Lake:8-Paul Lake:7    -0.3601111 -11.2902412  10.5700189 1.0000000
## Peter Lake:8-Paul Lake:7   16.7471111   5.9617574  27.5324648 0.0000827
## Paul Lake:9-Paul Lake:7     0.9891111 -15.5872518  17.5654741 1.0000000
## Peter Lake:9-Paul Lake:7   16.4723611  -1.6532090  34.5979312 0.1087387
## Paul Lake:8-Peter Lake:7  -25.0176928 -36.1073952 -13.9279904 0.0000000
## Peter Lake:8-Peter Lake:7  -7.9104706 -18.8575073   3.0365661 0.3778093
## Paul Lake:9-Peter Lake:7  -23.6684706 -40.3504838  -6.9864574 0.0004851
## Peter Lake:9-Peter Lake:7  -8.1852206 -26.4074611  10.0370199 0.9089776
## Peter Lake:8-Paul Lake:8   17.1072222   6.3218685  27.8925759 0.0000523
## Paul Lake:9-Paul Lake:8     1.3492222 -15.2271407  17.9255852 0.9999999
## Peter Lake:9-Paul Lake:8   16.8324722  -1.2930979  34.9580424 0.0926020
## Paul Lake:9-Peter Lake:8  -15.7580000 -32.2392597   0.7232597 0.0735733
## Peter Lake:9-Peter Lake:8  -0.2747500 -18.3133864  17.7638864 1.0000000
## Peter Lake:9-Paul Lake:9   15.4832500  -6.5132124  37.4797124 0.4163366
\end{verbatim}

\begin{Shaded}
\begin{Highlighting}[]
\NormalTok{TP.interaction <-}\StringTok{ }\KeywordTok{with}\NormalTok{(nutrient_seasonally, }\KeywordTok{interaction}\NormalTok{(lakename, month))}
\NormalTok{TP.anova}\FloatTok{.2}\NormalTok{way2 <-}\StringTok{ }\KeywordTok{aov}\NormalTok{(}\DataTypeTok{data =}\NormalTok{ nutrient_seasonally, tp_ug }\OperatorTok{~}\StringTok{ }\NormalTok{TP.interaction)}

\NormalTok{TP.anova}\FloatTok{.2}\NormalTok{way2.groups <-}\StringTok{ }\KeywordTok{HSD.test}\NormalTok{(TP.anova}\FloatTok{.2}\NormalTok{way2, }\StringTok{"TP.interaction"}\NormalTok{, }\DataTypeTok{group =} \OtherTok{TRUE}\NormalTok{)}
\NormalTok{TP.anova}\FloatTok{.2}\NormalTok{way2.groups }\CommentTok{# take all the p value from TukeyHSD, and group for me}
\end{Highlighting}
\end{Shaded}

\begin{verbatim}
## $statistics
##    MSerror  Df     Mean      CV
##   103.4055 119 19.07347 53.3141
## 
## $parameters
##    test         name.t ntr StudentizedRange alpha
##   Tukey TP.interaction  10         4.560262  0.05
## 
## $means
##                  tp_ug       std  r    Min    Max     Q25     Q50      Q75
## Paul Lake.5  11.474000  3.928545  6  7.001 17.090  8.1395 11.8885 13.53675
## Paul Lake.6  10.556118  4.416821 17  1.222 16.697  7.4430 10.6050 13.94600
## Paul Lake.7   9.746889  3.525120 18  4.501 21.763  7.8065  9.1555 10.65700
## Paul Lake.8   9.386778  1.478062 18  5.879 11.542  8.4495  9.6090 10.45050
## Paul Lake.9  10.736000  3.615978  5  6.592 16.281  8.9440 10.1920 11.67100
## Peter Lake.5 15.787571  2.719954  7 10.887 18.922 14.8915 15.5730 17.67400
## Peter Lake.6 28.357889 15.588507 18 10.974 53.388 14.7790 24.6840 41.13000
## Peter Lake.7 34.404471 18.285568 17 19.149 66.893 21.6640 24.2070 50.54900
## Peter Lake.8 26.494000  9.829596 19 14.551 49.757 21.2425 23.2250 27.99350
## Peter Lake.9 26.219250 10.814803  4 16.281 41.145 19.6845 23.7255 30.26025
## 
## $comparison
## NULL
## 
## $groups
##                  tp_ug groups
## Peter Lake.7 34.404471      a
## Peter Lake.6 28.357889     ab
## Peter Lake.8 26.494000    abc
## Peter Lake.9 26.219250   abcd
## Peter Lake.5 15.787571    bcd
## Paul Lake.5  11.474000     cd
## Paul Lake.9  10.736000     cd
## Paul Lake.6  10.556118      d
## Paul Lake.7   9.746889      d
## Paul Lake.8   9.386778      d
## 
## attr(,"class")
## [1] "group"
\end{verbatim}

\begin{enumerate}
\def\labelenumi{\arabic{enumi}.}
\setcounter{enumi}{6}
\item
  Create two plots, with TN (plot 1) or TP (plot 2) as the response
  variable and month and lake as the predictor variables. Hint: you may
  use some of the code you used for your visualization assignment.
  Assign groupings with letters, as determined from your tests. Adjust
  your axes, aesthetics, and color palettes in accordance with best data
  visualization practices.
\item
  Combine your plots with cowplot, with a common legend at the top and
  the two graphs stacked vertically. Your x axes should be formatted
  with the same breaks, such that you can remove the title and text of
  the top legend and retain just the bottom legend.
\end{enumerate}

\begin{Shaded}
\begin{Highlighting}[]
\CommentTok{#7}
\CommentTok{# plot1-TN}
\NormalTok{plot_TN <-}\StringTok{ }\KeywordTok{ggplot}\NormalTok{(nutrient_seasonally, }\KeywordTok{aes}\NormalTok{(}\DataTypeTok{x =}\NormalTok{ month, }\DataTypeTok{y =}\NormalTok{ tn_ug, }\DataTypeTok{color =}\NormalTok{ lakename)) }\OperatorTok{+}
\StringTok{  }\KeywordTok{geom_boxplot}\NormalTok{() }\OperatorTok{+}
\StringTok{  }\KeywordTok{labs}\NormalTok{(}\DataTypeTok{x=}\KeywordTok{expression}\NormalTok{(}\KeywordTok{paste}\NormalTok{(}\StringTok{"Month"}\NormalTok{))) }\OperatorTok{+}\StringTok{ }
\StringTok{  }\KeywordTok{labs}\NormalTok{(}\DataTypeTok{y=}\KeywordTok{expression}\NormalTok{(}\KeywordTok{paste}\NormalTok{(}\StringTok{"Total Nitrogen ("}\NormalTok{,mu,}\StringTok{"g/L)"}\NormalTok{))) }\OperatorTok{+}
\StringTok{  }\KeywordTok{labs}\NormalTok{(}\DataTypeTok{color=}\StringTok{"Lake Name"}\NormalTok{) }\OperatorTok{+}
\StringTok{  }\KeywordTok{stat_summary}\NormalTok{(}\DataTypeTok{geom =} \StringTok{"text"}\NormalTok{, }\DataTypeTok{fun.y =}\NormalTok{ max, }\DataTypeTok{vjust =} \DecValTok{-1}\NormalTok{, }\DataTypeTok{size =} \DecValTok{4}\NormalTok{,}
  \DataTypeTok{label =} \KeywordTok{c}\NormalTok{(}\StringTok{"a"}\NormalTok{, }\StringTok{"b"}\NormalTok{, }\StringTok{"a"}\NormalTok{, }\StringTok{"b"}\NormalTok{, }\StringTok{"a"}\NormalTok{, }\StringTok{"b"}\NormalTok{, }\StringTok{"a"}\NormalTok{,}\StringTok{"b"}\NormalTok{, }\StringTok{"a"}\NormalTok{, }\StringTok{"b"}\NormalTok{),}
  \DataTypeTok{position =} \KeywordTok{position_dodge}\NormalTok{(}\DataTypeTok{width=}\FloatTok{0.75}\NormalTok{)) }\OperatorTok{+}\StringTok{  }\CommentTok{# how do we determine the numer 0.75}
\StringTok{  }\KeywordTok{scale_color_brewer}\NormalTok{(}\DataTypeTok{palette =} \StringTok{"Set1"}\NormalTok{)}
  
\NormalTok{plot_TN}
\end{Highlighting}
\end{Shaded}

\begin{verbatim}
## Warning: Removed 23 rows containing non-finite values (stat_boxplot).
\end{verbatim}

\begin{verbatim}
## Warning: Removed 23 rows containing non-finite values (stat_summary).
\end{verbatim}

\includegraphics{Feng_A06_GLMs_Week1_files/figure-latex/unnamed-chunk-4-1.pdf}

\begin{Shaded}
\begin{Highlighting}[]
\CommentTok{# plot2-TP}
\NormalTok{plot_TP <-}\StringTok{ }\KeywordTok{ggplot}\NormalTok{(nutrient_seasonally, }\KeywordTok{aes}\NormalTok{(}\DataTypeTok{x =}\NormalTok{ month, }\DataTypeTok{y =}\NormalTok{ tp_ug, }\DataTypeTok{color =}\NormalTok{ lakename)) }\OperatorTok{+}\StringTok{ }
\StringTok{  }\KeywordTok{geom_boxplot}\NormalTok{() }\OperatorTok{+}
\StringTok{  }\KeywordTok{labs}\NormalTok{(}\DataTypeTok{x=}\KeywordTok{expression}\NormalTok{(}\KeywordTok{paste}\NormalTok{(}\StringTok{"Month"}\NormalTok{))) }\OperatorTok{+}\StringTok{ }
\StringTok{  }\KeywordTok{labs}\NormalTok{(}\DataTypeTok{y=}\KeywordTok{expression}\NormalTok{(}\KeywordTok{paste}\NormalTok{(}\StringTok{"Total Phosphorus ("}\NormalTok{,mu,}\StringTok{"g/L)"}\NormalTok{)))}\OperatorTok{+}
\StringTok{  }\KeywordTok{labs}\NormalTok{(}\DataTypeTok{color=}\StringTok{"Lake Name"}\NormalTok{)}\OperatorTok{+}
\StringTok{  }\KeywordTok{stat_summary}\NormalTok{(}\DataTypeTok{geom =} \StringTok{"text"}\NormalTok{, }\DataTypeTok{fun.y =}\NormalTok{ max, }\DataTypeTok{vjust =} \DecValTok{-1}\NormalTok{, }\DataTypeTok{size =} \DecValTok{4}\NormalTok{,}
  \DataTypeTok{label =} \KeywordTok{c}\NormalTok{(}\StringTok{"bcd"}\NormalTok{,}\StringTok{"cd"}\NormalTok{, }\StringTok{"ab"}\NormalTok{, }\StringTok{"d"}\NormalTok{, }\StringTok{"a"}\NormalTok{, }\StringTok{"d"}\NormalTok{, }\StringTok{"abc"}\NormalTok{, }\StringTok{"d"}\NormalTok{, }\StringTok{"abcd"}\NormalTok{,  }\StringTok{"cd"}\NormalTok{),}
  \DataTypeTok{position =} \KeywordTok{position_dodge}\NormalTok{(}\DataTypeTok{width=}\FloatTok{0.75}\NormalTok{)) }\OperatorTok{+}\StringTok{    }\CommentTok{# how do we determine the numer 0.75}
\StringTok{  }\KeywordTok{scale_color_brewer}\NormalTok{(}\DataTypeTok{palette =} \StringTok{"Set1"}\NormalTok{)}

\NormalTok{plot_TP}
\end{Highlighting}
\end{Shaded}

\begin{verbatim}
## Warning: Removed 1 rows containing non-finite values (stat_boxplot).
\end{verbatim}

\begin{verbatim}
## Warning: Removed 1 rows containing non-finite values (stat_summary).
\end{verbatim}

\includegraphics{Feng_A06_GLMs_Week1_files/figure-latex/unnamed-chunk-4-2.pdf}

\begin{Shaded}
\begin{Highlighting}[]
\CommentTok{#8}
\NormalTok{CombinedPlot <-}\StringTok{ }\KeywordTok{plot_grid}\NormalTok{(plot_TN}\OperatorTok{+}\KeywordTok{theme}\NormalTok{(}\DataTypeTok{legend.position=}\StringTok{"top"}\NormalTok{),}
\NormalTok{                          plot_TP}\OperatorTok{+}\KeywordTok{theme}\NormalTok{(}\DataTypeTok{legend.position=}\StringTok{"none"}\NormalTok{), }
                          \DataTypeTok{nrow=}\DecValTok{2}\NormalTok{, }\DataTypeTok{align =} \StringTok{'v'}\NormalTok{, }\DataTypeTok{axis =} \StringTok{'l'}\NormalTok{, }\DataTypeTok{labels=}\KeywordTok{c}\NormalTok{(}\StringTok{'A'}\NormalTok{, }\StringTok{'B'}\NormalTok{))   }
\end{Highlighting}
\end{Shaded}

\begin{verbatim}
## Warning: Removed 23 rows containing non-finite values (stat_boxplot).
\end{verbatim}

\begin{verbatim}
## Warning: Removed 23 rows containing non-finite values (stat_summary).
\end{verbatim}

\begin{verbatim}
## Warning: Removed 1 rows containing non-finite values (stat_boxplot).
\end{verbatim}

\begin{verbatim}
## Warning: Removed 1 rows containing non-finite values (stat_summary).
\end{verbatim}

\begin{Shaded}
\begin{Highlighting}[]
\KeywordTok{print}\NormalTok{(CombinedPlot)}
\end{Highlighting}
\end{Shaded}

\includegraphics{Feng_A06_GLMs_Week1_files/figure-latex/unnamed-chunk-4-3.pdf}


\end{document}
