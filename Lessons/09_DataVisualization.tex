\documentclass[]{article}
\usepackage{lmodern}
\usepackage{amssymb,amsmath}
\usepackage{ifxetex,ifluatex}
\usepackage{fixltx2e} % provides \textsubscript
\ifnum 0\ifxetex 1\fi\ifluatex 1\fi=0 % if pdftex
  \usepackage[T1]{fontenc}
  \usepackage[utf8]{inputenc}
\else % if luatex or xelatex
  \ifxetex
    \usepackage{mathspec}
  \else
    \usepackage{fontspec}
  \fi
  \defaultfontfeatures{Ligatures=TeX,Scale=MatchLowercase}
\fi
% use upquote if available, for straight quotes in verbatim environments
\IfFileExists{upquote.sty}{\usepackage{upquote}}{}
% use microtype if available
\IfFileExists{microtype.sty}{%
\usepackage{microtype}
\UseMicrotypeSet[protrusion]{basicmath} % disable protrusion for tt fonts
}{}
\usepackage[margin=2.54cm]{geometry}
\usepackage{hyperref}
\hypersetup{unicode=true,
            pdftitle={9: Data Visualization Advanced},
            pdfauthor={Environmental Data Analytics \textbar{} Kateri Salk},
            pdfborder={0 0 0},
            breaklinks=true}
\urlstyle{same}  % don't use monospace font for urls
\usepackage{color}
\usepackage{fancyvrb}
\newcommand{\VerbBar}{|}
\newcommand{\VERB}{\Verb[commandchars=\\\{\}]}
\DefineVerbatimEnvironment{Highlighting}{Verbatim}{commandchars=\\\{\}}
% Add ',fontsize=\small' for more characters per line
\usepackage{framed}
\definecolor{shadecolor}{RGB}{248,248,248}
\newenvironment{Shaded}{\begin{snugshade}}{\end{snugshade}}
\newcommand{\AlertTok}[1]{\textcolor[rgb]{0.94,0.16,0.16}{#1}}
\newcommand{\AnnotationTok}[1]{\textcolor[rgb]{0.56,0.35,0.01}{\textbf{\textit{#1}}}}
\newcommand{\AttributeTok}[1]{\textcolor[rgb]{0.77,0.63,0.00}{#1}}
\newcommand{\BaseNTok}[1]{\textcolor[rgb]{0.00,0.00,0.81}{#1}}
\newcommand{\BuiltInTok}[1]{#1}
\newcommand{\CharTok}[1]{\textcolor[rgb]{0.31,0.60,0.02}{#1}}
\newcommand{\CommentTok}[1]{\textcolor[rgb]{0.56,0.35,0.01}{\textit{#1}}}
\newcommand{\CommentVarTok}[1]{\textcolor[rgb]{0.56,0.35,0.01}{\textbf{\textit{#1}}}}
\newcommand{\ConstantTok}[1]{\textcolor[rgb]{0.00,0.00,0.00}{#1}}
\newcommand{\ControlFlowTok}[1]{\textcolor[rgb]{0.13,0.29,0.53}{\textbf{#1}}}
\newcommand{\DataTypeTok}[1]{\textcolor[rgb]{0.13,0.29,0.53}{#1}}
\newcommand{\DecValTok}[1]{\textcolor[rgb]{0.00,0.00,0.81}{#1}}
\newcommand{\DocumentationTok}[1]{\textcolor[rgb]{0.56,0.35,0.01}{\textbf{\textit{#1}}}}
\newcommand{\ErrorTok}[1]{\textcolor[rgb]{0.64,0.00,0.00}{\textbf{#1}}}
\newcommand{\ExtensionTok}[1]{#1}
\newcommand{\FloatTok}[1]{\textcolor[rgb]{0.00,0.00,0.81}{#1}}
\newcommand{\FunctionTok}[1]{\textcolor[rgb]{0.00,0.00,0.00}{#1}}
\newcommand{\ImportTok}[1]{#1}
\newcommand{\InformationTok}[1]{\textcolor[rgb]{0.56,0.35,0.01}{\textbf{\textit{#1}}}}
\newcommand{\KeywordTok}[1]{\textcolor[rgb]{0.13,0.29,0.53}{\textbf{#1}}}
\newcommand{\NormalTok}[1]{#1}
\newcommand{\OperatorTok}[1]{\textcolor[rgb]{0.81,0.36,0.00}{\textbf{#1}}}
\newcommand{\OtherTok}[1]{\textcolor[rgb]{0.56,0.35,0.01}{#1}}
\newcommand{\PreprocessorTok}[1]{\textcolor[rgb]{0.56,0.35,0.01}{\textit{#1}}}
\newcommand{\RegionMarkerTok}[1]{#1}
\newcommand{\SpecialCharTok}[1]{\textcolor[rgb]{0.00,0.00,0.00}{#1}}
\newcommand{\SpecialStringTok}[1]{\textcolor[rgb]{0.31,0.60,0.02}{#1}}
\newcommand{\StringTok}[1]{\textcolor[rgb]{0.31,0.60,0.02}{#1}}
\newcommand{\VariableTok}[1]{\textcolor[rgb]{0.00,0.00,0.00}{#1}}
\newcommand{\VerbatimStringTok}[1]{\textcolor[rgb]{0.31,0.60,0.02}{#1}}
\newcommand{\WarningTok}[1]{\textcolor[rgb]{0.56,0.35,0.01}{\textbf{\textit{#1}}}}
\usepackage{graphicx,grffile}
\makeatletter
\def\maxwidth{\ifdim\Gin@nat@width>\linewidth\linewidth\else\Gin@nat@width\fi}
\def\maxheight{\ifdim\Gin@nat@height>\textheight\textheight\else\Gin@nat@height\fi}
\makeatother
% Scale images if necessary, so that they will not overflow the page
% margins by default, and it is still possible to overwrite the defaults
% using explicit options in \includegraphics[width, height, ...]{}
\setkeys{Gin}{width=\maxwidth,height=\maxheight,keepaspectratio}
\IfFileExists{parskip.sty}{%
\usepackage{parskip}
}{% else
\setlength{\parindent}{0pt}
\setlength{\parskip}{6pt plus 2pt minus 1pt}
}
\setlength{\emergencystretch}{3em}  % prevent overfull lines
\providecommand{\tightlist}{%
  \setlength{\itemsep}{0pt}\setlength{\parskip}{0pt}}
\setcounter{secnumdepth}{0}
% Redefines (sub)paragraphs to behave more like sections
\ifx\paragraph\undefined\else
\let\oldparagraph\paragraph
\renewcommand{\paragraph}[1]{\oldparagraph{#1}\mbox{}}
\fi
\ifx\subparagraph\undefined\else
\let\oldsubparagraph\subparagraph
\renewcommand{\subparagraph}[1]{\oldsubparagraph{#1}\mbox{}}
\fi

%%% Use protect on footnotes to avoid problems with footnotes in titles
\let\rmarkdownfootnote\footnote%
\def\footnote{\protect\rmarkdownfootnote}

%%% Change title format to be more compact
\usepackage{titling}

% Create subtitle command for use in maketitle
\providecommand{\subtitle}[1]{
  \posttitle{
    \begin{center}\large#1\end{center}
    }
}

\setlength{\droptitle}{-2em}

  \title{9: Data Visualization Advanced}
    \pretitle{\vspace{\droptitle}\centering\huge}
  \posttitle{\par}
    \author{Environmental Data Analytics \textbar{} Kateri Salk}
    \preauthor{\centering\large\emph}
  \postauthor{\par}
      \predate{\centering\large\emph}
  \postdate{\par}
    \date{Spring 2020}


\begin{document}
\maketitle

\hypertarget{lesson-objectives}{%
\subsection{LESSON OBJECTIVES}\label{lesson-objectives}}

\begin{enumerate}
\def\labelenumi{\arabic{enumi}.}
\tightlist
\item
  Perform advanced edits on ggplot objects to follow best practices for
  data visualization
\item
  Troubleshoot visualization challenges
\end{enumerate}

\hypertarget{set-up-your-data-analysis-session}{%
\subsection{SET UP YOUR DATA ANALYSIS
SESSION}\label{set-up-your-data-analysis-session}}

\begin{Shaded}
\begin{Highlighting}[]
\KeywordTok{getwd}\NormalTok{()}
\end{Highlighting}
\end{Shaded}

\begin{verbatim}
## [1] "/Users/ethel/Desktop/Environ 872/Environmental_Data_Analytics_2020"
\end{verbatim}

\begin{Shaded}
\begin{Highlighting}[]
\KeywordTok{library}\NormalTok{(tidyverse)}

\NormalTok{PeterPaul.chem.nutrients <-}\StringTok{ }
\StringTok{  }\KeywordTok{read.csv}\NormalTok{(}\StringTok{"./Data/Processed/NTL-LTER_Lake_Chemistry_Nutrients_PeterPaul_Processed.csv"}\NormalTok{)}
\NormalTok{PeterPaul.chem.nutrients.gathered <-}
\StringTok{  }\KeywordTok{read.csv}\NormalTok{(}\StringTok{"./Data/Processed/NTL-LTER_Lake_Nutrients_PeterPaulGathered_Processed.csv"}\NormalTok{)}
\NormalTok{EPAair <-}\StringTok{ }\KeywordTok{read.csv}\NormalTok{(}\StringTok{"./Data/Processed/EPAair_O3_PM25_NC1819_Processed.csv"}\NormalTok{)}

\NormalTok{EPAair}\OperatorTok{$}\NormalTok{Date <-}\StringTok{ }\KeywordTok{as.Date}\NormalTok{(EPAair}\OperatorTok{$}\NormalTok{Date, }\DataTypeTok{format =} \StringTok{"%Y-%m-%d"}\NormalTok{)}
\NormalTok{PeterPaul.chem.nutrients}\OperatorTok{$}\NormalTok{sampledate <-}\StringTok{ }\KeywordTok{as.Date}\NormalTok{(}
\NormalTok{  PeterPaul.chem.nutrients}\OperatorTok{$}\NormalTok{sampledate, }\DataTypeTok{format =} \StringTok{"%Y-%m-%d"}\NormalTok{)}
\NormalTok{PeterPaul.chem.nutrients.gathered}\OperatorTok{$}\NormalTok{sampledate <-}\StringTok{ }\KeywordTok{as.Date}\NormalTok{(}
\NormalTok{  PeterPaul.chem.nutrients.gathered}\OperatorTok{$}\NormalTok{sampledate, }\DataTypeTok{format =} \StringTok{"%Y-%m-%d"}\NormalTok{)}
\end{Highlighting}
\end{Shaded}

\hypertarget{themes}{%
\subsubsection{Themes}\label{themes}}

Often, we will want to change multiple visual aspects of a plot. Ggplot
comes with pre-built themes that will adjust components of plots if you
call that theme.

\begin{Shaded}
\begin{Highlighting}[]
\NormalTok{O3plot <-}\StringTok{ }\KeywordTok{ggplot}\NormalTok{(EPAair) }\OperatorTok{+}
\StringTok{  }\KeywordTok{geom_point}\NormalTok{(}\KeywordTok{aes}\NormalTok{(}\DataTypeTok{x =}\NormalTok{ Date, }\DataTypeTok{y =}\NormalTok{ Ozone)) }
\KeywordTok{print}\NormalTok{(O3plot)}
\end{Highlighting}
\end{Shaded}

\includegraphics{09_DataVisualization_files/figure-latex/unnamed-chunk-2-1.pdf}

\begin{Shaded}
\begin{Highlighting}[]
\NormalTok{O3plot1 <-}\StringTok{ }\KeywordTok{ggplot}\NormalTok{(EPAair) }\OperatorTok{+}
\StringTok{  }\KeywordTok{geom_point}\NormalTok{(}\KeywordTok{aes}\NormalTok{(}\DataTypeTok{x =}\NormalTok{ Date, }\DataTypeTok{y =}\NormalTok{ Ozone)) }\OperatorTok{+}
\StringTok{  }\KeywordTok{theme_gray}\NormalTok{()}
\KeywordTok{print}\NormalTok{(O3plot1)}
\end{Highlighting}
\end{Shaded}

\includegraphics{09_DataVisualization_files/figure-latex/unnamed-chunk-2-2.pdf}

\begin{Shaded}
\begin{Highlighting}[]
\NormalTok{O3plot2 <-}\StringTok{ }\KeywordTok{ggplot}\NormalTok{(EPAair) }\OperatorTok{+}
\StringTok{  }\KeywordTok{geom_point}\NormalTok{(}\KeywordTok{aes}\NormalTok{(}\DataTypeTok{x =}\NormalTok{ Date, }\DataTypeTok{y =}\NormalTok{ Ozone)) }\OperatorTok{+}
\StringTok{  }\KeywordTok{theme_bw}\NormalTok{()  }\CommentTok{#theme is background}
\KeywordTok{print}\NormalTok{(O3plot2)}
\end{Highlighting}
\end{Shaded}

\includegraphics{09_DataVisualization_files/figure-latex/unnamed-chunk-2-3.pdf}

\begin{Shaded}
\begin{Highlighting}[]
\NormalTok{O3plot3 <-}\StringTok{ }\KeywordTok{ggplot}\NormalTok{(EPAair) }\OperatorTok{+}
\StringTok{  }\KeywordTok{geom_point}\NormalTok{(}\KeywordTok{aes}\NormalTok{(}\DataTypeTok{x =}\NormalTok{ Date, }\DataTypeTok{y =}\NormalTok{ Ozone)) }\OperatorTok{+}
\StringTok{  }\KeywordTok{theme_light}\NormalTok{()}
\KeywordTok{print}\NormalTok{(O3plot3)}
\end{Highlighting}
\end{Shaded}

\includegraphics{09_DataVisualization_files/figure-latex/unnamed-chunk-2-4.pdf}

\begin{Shaded}
\begin{Highlighting}[]
\NormalTok{O3plot4 <-}\StringTok{ }\KeywordTok{ggplot}\NormalTok{(EPAair) }\OperatorTok{+}
\StringTok{  }\KeywordTok{geom_point}\NormalTok{(}\KeywordTok{aes}\NormalTok{(}\DataTypeTok{x =}\NormalTok{ Date, }\DataTypeTok{y =}\NormalTok{ Ozone)) }\OperatorTok{+}
\StringTok{  }\KeywordTok{theme_classic}\NormalTok{() }\CommentTok{#bakeground no grid}
\KeywordTok{print}\NormalTok{(O3plot4)}
\end{Highlighting}
\end{Shaded}

\includegraphics{09_DataVisualization_files/figure-latex/unnamed-chunk-2-5.pdf}

Notice that some aspects of your graph have not been adjusted,
including:

\begin{itemize}
\tightlist
\item
  text size
\item
  axis label colors
\item
  legend position and justification
\end{itemize}

If you would like to set a common theme across all plots in your
analysis session, you may define a theme and call up that theme for each
graph. This eliminates the need to add multiple lines of code in each
plot.

\begin{Shaded}
\begin{Highlighting}[]
\NormalTok{mytheme <-}\StringTok{ }\KeywordTok{theme_classic}\NormalTok{(}\DataTypeTok{base_size =} \DecValTok{14}\NormalTok{) }\OperatorTok{+}
\StringTok{  }\KeywordTok{theme}\NormalTok{(}\DataTypeTok{axis.text =} \KeywordTok{element_text}\NormalTok{(}\DataTypeTok{color =} \StringTok{"DarkBlue"}\NormalTok{), }
        \DataTypeTok{legend.position =} \StringTok{"top"}\NormalTok{) }\CommentTok{#alternative: legend.position + legend.justification}

\CommentTok{# options: call the theme in each plot or set the theme at the start. }

\NormalTok{O3plot5 <-}\StringTok{ }\KeywordTok{ggplot}\NormalTok{(EPAair) }\OperatorTok{+}
\StringTok{  }\KeywordTok{geom_point}\NormalTok{(}\KeywordTok{aes}\NormalTok{(}\DataTypeTok{x =}\NormalTok{ Date, }\DataTypeTok{y =}\NormalTok{ Ozone)) }\OperatorTok{+}
\StringTok{  }\NormalTok{mytheme  }\CommentTok{# setting on above (seperate)}
\KeywordTok{print}\NormalTok{(O3plot5)  }
\end{Highlighting}
\end{Shaded}

\includegraphics{09_DataVisualization_files/figure-latex/unnamed-chunk-3-1.pdf}

\begin{Shaded}
\begin{Highlighting}[]
\KeywordTok{theme_set}\NormalTok{(mytheme) }\CommentTok{#all the plot gonna be this theme and do not add the theme in plot code}

\NormalTok{O3plot6 <-}\StringTok{ }\KeywordTok{ggplot}\NormalTok{(EPAair) }\OperatorTok{+}
\StringTok{  }\KeywordTok{geom_point}\NormalTok{(}\KeywordTok{aes}\NormalTok{(}\DataTypeTok{x =}\NormalTok{ Date, }\DataTypeTok{y =}\NormalTok{ Ozone))}
\KeywordTok{print}\NormalTok{(O3plot6)  }
\end{Highlighting}
\end{Shaded}

\includegraphics{09_DataVisualization_files/figure-latex/unnamed-chunk-3-2.pdf}

\hypertarget{adjusting-multiple-components-of-your-plots}{%
\subsubsection{Adjusting multiple components of your
plots}\label{adjusting-multiple-components-of-your-plots}}

While the theme allows us to set multiple aspects of plots, ggplot
allows us to adjust other parts of plots outside of the theme.

\begin{Shaded}
\begin{Highlighting}[]
\NormalTok{O3plot7 <-}\StringTok{ }\KeywordTok{ggplot}\NormalTok{(EPAair, }\KeywordTok{aes}\NormalTok{(}\DataTypeTok{x =}\NormalTok{ Date, }\DataTypeTok{y =}\NormalTok{ Ozone)) }\OperatorTok{+}\StringTok{ }\CommentTok{# add layers below}
\StringTok{  }\KeywordTok{geom_hline}\NormalTok{(}\DataTypeTok{yintercept =} \DecValTok{50}\NormalTok{, }\DataTypeTok{lty =} \DecValTok{2}\NormalTok{) }\OperatorTok{+}\StringTok{   }\CommentTok{# horizational line}
\StringTok{  }\KeywordTok{geom_hline}\NormalTok{(}\DataTypeTok{yintercept =} \DecValTok{100}\NormalTok{, }\DataTypeTok{lty =} \DecValTok{2}\NormalTok{) }\OperatorTok{+}
\StringTok{  }\KeywordTok{geom_point}\NormalTok{(}\DataTypeTok{alpha =} \FloatTok{0.5}\NormalTok{, }\DataTypeTok{size =} \FloatTok{1.5}\NormalTok{) }\OperatorTok{+}\StringTok{ }\CommentTok{#ggplot does thing in order, the point gonna be below than line}
\StringTok{                                        }\CommentTok{#alpha = 0.5,transparent 50%}
\StringTok{  }\KeywordTok{geom_text}\NormalTok{(}\DataTypeTok{x =} \KeywordTok{as.Date}\NormalTok{(}\StringTok{"2020-01-01"}\NormalTok{), }\DataTypeTok{y =} \DecValTok{45}\NormalTok{, }\DataTypeTok{label =} \StringTok{"good"}\NormalTok{, }\DataTypeTok{hjust =} \DecValTok{1}\NormalTok{, }\DataTypeTok{fontface =} \StringTok{"bold"}\NormalTok{) }\OperatorTok{+}
\StringTok{  }\KeywordTok{geom_text}\NormalTok{(}\DataTypeTok{x =} \KeywordTok{as.Date}\NormalTok{(}\StringTok{"2020-01-01"}\NormalTok{), }\DataTypeTok{y =} \DecValTok{95}\NormalTok{, }\DataTypeTok{label =} \StringTok{"moderate"}\NormalTok{, }\DataTypeTok{hjust =} \DecValTok{1}\NormalTok{, }\DataTypeTok{fontface =} \StringTok{"bold"}\NormalTok{) }\OperatorTok{+}
\StringTok{  }\KeywordTok{geom_text}\NormalTok{(}\DataTypeTok{x =} \KeywordTok{as.Date}\NormalTok{(}\StringTok{"2020-01-01"}\NormalTok{), }\DataTypeTok{y =} \DecValTok{120}\NormalTok{, }\DataTypeTok{label =} \StringTok{"unhealthy (sensitive groups)"}\NormalTok{, }\DataTypeTok{hjust =} \DecValTok{1}\NormalTok{, }\DataTypeTok{fontface =} \StringTok{"bold"}\NormalTok{) }\OperatorTok{+}
\StringTok{  }\KeywordTok{scale_x_date}\NormalTok{(}\DataTypeTok{limits =} \KeywordTok{as.Date}\NormalTok{(}\KeywordTok{c}\NormalTok{(}\StringTok{"2018-01-01"}\NormalTok{, }\StringTok{"2019-12-31"}\NormalTok{)),  }\CommentTok{#change the data scale}
    \DataTypeTok{date_breaks =} \StringTok{"2 months"}\NormalTok{, }\DataTypeTok{date_labels =} \StringTok{"%b %y"}\NormalTok{) }\OperatorTok{+}\StringTok{  }\CommentTok{#date_labels is re-format }
\StringTok{  }\KeywordTok{ylab}\NormalTok{(}\KeywordTok{expression}\NormalTok{(}\StringTok{"O"}\NormalTok{[}\DecValTok{3}\NormalTok{]}\OperatorTok{*}\StringTok{ " AQI Value"}\NormalTok{)) }\OperatorTok{+}\StringTok{  }\CommentTok{#[3] is subscript}
\StringTok{  }\KeywordTok{theme}\NormalTok{(}\DataTypeTok{axis.text.x =} \KeywordTok{element_text}\NormalTok{(}\DataTypeTok{angle =} \DecValTok{45}\NormalTok{,  }\DataTypeTok{hjust =} \DecValTok{1}\NormalTok{)) }\CommentTok{#angle = 45 is 45% degrees}
\KeywordTok{print}\NormalTok{(O3plot7)  }
\end{Highlighting}
\end{Shaded}

\includegraphics{09_DataVisualization_files/figure-latex/unnamed-chunk-4-1.pdf}

\hypertarget{color-palettes}{%
\subsubsection{Color palettes}\label{color-palettes}}

Color palettes are an effective way to communicate additional aspects of
our data, often illustrating a third categorical or continuous variable
in addition to the variables on the x and y axes. A few rules for
choosing colors:

\begin{itemize}
\tightlist
\item
  Consider if your plot needs to be viewed in black and white. If so,
  choose a sequential palette with varying color intensity.
\item
  Choose a palette that is color-blind friendly
\item
  Maximize contrast (e.g., no pale colors on a white background)
\item
  Diverging color palettes should be used for diverging values (e.g.,
  warm-to-cool works well for values on a scale encompassing negative
  and positive values)
\end{itemize}

Does your color palette communicate additional and necessary
information? If the answer is no, then you might consider removing it
and going with a single color. Common instances of superfluous or
redundant color palettes include:

\begin{itemize}
\tightlist
\item
  Color that duplicates an axis
\item
  Color that distinguishes categories when labels already exist
  (exception: if category colors repeat throughout a series of
  interrelated visualizations and help the reader build a frame of
  reference across a report)
\item
  Color that reduces the conciseness of a plot
\end{itemize}

Perception is key! Choose palettes that are visually pleasing and will
communicate what you are hoping your audience to perceive.

RColorBrewer (package)

\begin{itemize}
\tightlist
\item
  \url{http://colorbrewer2.org}
\item
  \url{https://moderndata.plot.ly/create-colorful-graphs-in-r-with-rcolorbrewer-and-plotly/}
\end{itemize}

viridis and viridisLite (packages)

\begin{itemize}
\tightlist
\item
  \url{https://cran.r-project.org/web/packages/viridis/vignettes/intro-to-viridis.html}
\item
  \url{https://ggplot2.tidyverse.org/reference/scale_viridis.html}
\end{itemize}

colorRamp (function; comes with base R as part of the grDevices package)

\begin{itemize}
\tightlist
\item
  \url{https://bookdown.org/rdpeng/exdata/plotting-and-color-in-r.html\#colorramp}
\end{itemize}

LaCroixColoR (package)

\begin{itemize}
\tightlist
\item
  \url{https://github.com/johannesbjork/LaCroixColoR}
\end{itemize}

wesanderson (package)

\begin{itemize}
\tightlist
\item
  \url{https://github.com/karthik/wesanderson}
\end{itemize}

nationalparkcolors (package)

\begin{itemize}
\tightlist
\item
  \url{https://github.com/katiejolly/nationalparkcolors}
\end{itemize}

\begin{Shaded}
\begin{Highlighting}[]
\CommentTok{#install.packages("viridis")}
\CommentTok{#install.packages("RColorBrewer")}
\CommentTok{#install.packages("colormap")}
\KeywordTok{library}\NormalTok{(viridis)}
\end{Highlighting}
\end{Shaded}

\begin{verbatim}
## Loading required package: viridisLite
\end{verbatim}

\begin{Shaded}
\begin{Highlighting}[]
\KeywordTok{library}\NormalTok{(RColorBrewer)}
\KeywordTok{library}\NormalTok{(colormap)}

\NormalTok{scales}\OperatorTok{::}\KeywordTok{show_col}\NormalTok{(}\KeywordTok{colormap}\NormalTok{(}\DataTypeTok{colormap =}\NormalTok{ colormaps}\OperatorTok{$}\NormalTok{viridis, }\DataTypeTok{nshades =} \DecValTok{16}\NormalTok{))}
\end{Highlighting}
\end{Shaded}

\includegraphics{09_DataVisualization_files/figure-latex/unnamed-chunk-5-1.pdf}

\begin{Shaded}
\begin{Highlighting}[]
\NormalTok{scales}\OperatorTok{::}\KeywordTok{show_col}\NormalTok{(}\KeywordTok{colormap}\NormalTok{(}\DataTypeTok{colormap =}\NormalTok{ colormaps}\OperatorTok{$}\NormalTok{inferno, }\DataTypeTok{nshades =} \DecValTok{16}\NormalTok{))}
\end{Highlighting}
\end{Shaded}

\includegraphics{09_DataVisualization_files/figure-latex/unnamed-chunk-5-2.pdf}

\begin{Shaded}
\begin{Highlighting}[]
\NormalTok{scales}\OperatorTok{::}\KeywordTok{show_col}\NormalTok{(}\KeywordTok{colormap}\NormalTok{(}\DataTypeTok{colormap =}\NormalTok{ colormaps}\OperatorTok{$}\NormalTok{magma, }\DataTypeTok{nshades =} \DecValTok{16}\NormalTok{))}
\end{Highlighting}
\end{Shaded}

\includegraphics{09_DataVisualization_files/figure-latex/unnamed-chunk-5-3.pdf}

\begin{Shaded}
\begin{Highlighting}[]
\KeywordTok{display.brewer.all}\NormalTok{(}\DataTypeTok{n =} \DecValTok{9}\NormalTok{)}
\end{Highlighting}
\end{Shaded}

\includegraphics{09_DataVisualization_files/figure-latex/unnamed-chunk-5-4.pdf}

\begin{Shaded}
\begin{Highlighting}[]
\NormalTok{NvsP <-}
\StringTok{  }\KeywordTok{ggplot}\NormalTok{(PeterPaul.chem.nutrients, }\KeywordTok{aes}\NormalTok{(}\DataTypeTok{x =}\NormalTok{ tp_ug, }\DataTypeTok{y =}\NormalTok{ tn_ug, }\DataTypeTok{color =}\NormalTok{ depth, }\DataTypeTok{shape =}\NormalTok{ lakename)) }\OperatorTok{+}
\StringTok{  }\KeywordTok{geom_point}\NormalTok{() }
\KeywordTok{print}\NormalTok{(NvsP)}
\end{Highlighting}
\end{Shaded}

\includegraphics{09_DataVisualization_files/figure-latex/unnamed-chunk-5-5.pdf}

\begin{Shaded}
\begin{Highlighting}[]
\CommentTok{# let's first make the plot look better.}
\CommentTok{# change your axis labels to reflect TN and TP in micrograms per liter.}
\CommentTok{# change your legend labels}
\NormalTok{NvsP2 <-}
\StringTok{  }\KeywordTok{ggplot}\NormalTok{(PeterPaul.chem.nutrients, }\KeywordTok{aes}\NormalTok{(}\DataTypeTok{x =}\NormalTok{ tp_ug, }\DataTypeTok{y =}\NormalTok{ tn_ug, }\DataTypeTok{color =}\NormalTok{ depth, }\DataTypeTok{shape =}\NormalTok{ lakename)) }\OperatorTok{+}
\StringTok{  }\KeywordTok{geom_point}\NormalTok{(}\DataTypeTok{alpha =} \FloatTok{0.7}\NormalTok{, }\DataTypeTok{size =} \FloatTok{2.5}\NormalTok{) }\OperatorTok{+}\StringTok{  }\CommentTok{# aparency and poit size}
\StringTok{  }\KeywordTok{labs}\NormalTok{(}\DataTypeTok{x=}\KeywordTok{expression}\NormalTok{(}\KeywordTok{paste}\NormalTok{(}\StringTok{"TP("}\NormalTok{,mu,}\StringTok{"g/L)"}\NormalTok{))) }\OperatorTok{+}\StringTok{ }\CommentTok{# change your legend labels here #mu greek number}
\StringTok{  }\KeywordTok{labs}\NormalTok{(}\DataTypeTok{y=}\KeywordTok{expression}\NormalTok{(}\KeywordTok{paste}\NormalTok{(}\StringTok{"TN("}\NormalTok{,mu,}\StringTok{"g/L)"}\NormalTok{)))}\OperatorTok{+}
\StringTok{  }\KeywordTok{labs}\NormalTok{(}\DataTypeTok{color=}\StringTok{"depth(m)"}\NormalTok{)}\OperatorTok{+}\StringTok{ }\CommentTok{#color is for the legend}
\StringTok{  }\KeywordTok{scale_shape_manual}\NormalTok{(}\DataTypeTok{values =} \KeywordTok{c}\NormalTok{(}\DecValTok{15}\NormalTok{, }\DecValTok{17}\NormalTok{)) }\OperatorTok{+}\StringTok{  }\CommentTok{# google R shapes, it is  points symbols }
\StringTok{  }\KeywordTok{scale_color_distiller}\NormalTok{(}\DataTypeTok{palette =} \StringTok{"Blues"}\NormalTok{, }\DataTypeTok{direction =} \DecValTok{1}\NormalTok{) }\OperatorTok{+}\StringTok{ }\CommentTok{# use scale_color_brewer for discrete variables}
\StringTok{  }\CommentTok{#scale_color_viridis(option = "magma", direction = -1, end=0.8) + #direction=-1: color from light to dark}
\StringTok{                                                                   }\CommentTok{#end=0.8 Change the gray value at the low and the high ends of the palette}
\StringTok{  }\KeywordTok{theme}\NormalTok{(}\DataTypeTok{legend.position =} \StringTok{"right"}\NormalTok{, }
        \DataTypeTok{legend.text =} \KeywordTok{element_text}\NormalTok{(}\DataTypeTok{size =} \DecValTok{12}\NormalTok{), }\DataTypeTok{legend.title =} \KeywordTok{element_text}\NormalTok{(}\DataTypeTok{size =} \DecValTok{12}\NormalTok{))}
\KeywordTok{print}\NormalTok{(NvsP2)}
\end{Highlighting}
\end{Shaded}

\includegraphics{09_DataVisualization_files/figure-latex/unnamed-chunk-5-6.pdf}

\begin{Shaded}
\begin{Highlighting}[]
\CommentTok{# change your y axis label to list concentration in micrograms per liter}
\CommentTok{# remove your x axis label}
\CommentTok{# change labels for nutrients in the legend}
\CommentTok{# try out the different color palette options and choose one (or edit)}
\NormalTok{Nutrientplot <-}
\StringTok{  }\KeywordTok{ggplot}\NormalTok{(PeterPaul.chem.nutrients.gathered, }\KeywordTok{aes}\NormalTok{(}\DataTypeTok{x =}\NormalTok{ lakename, }\DataTypeTok{y =}\NormalTok{ concentration, }\DataTypeTok{color =}\NormalTok{ nutrient)) }\OperatorTok{+}
\StringTok{  }\KeywordTok{geom_boxplot}\NormalTok{() }\OperatorTok{+}
\StringTok{  }\KeywordTok{labs}\NormalTok{(}\DataTypeTok{y=}\KeywordTok{expression}\NormalTok{(}\KeywordTok{paste}\NormalTok{(}\StringTok{"Concentration (mg/L)"}\NormalTok{))) }\OperatorTok{+}\StringTok{ }\CommentTok{# change your legend labels here}
\StringTok{  }\KeywordTok{labs}\NormalTok{(}\DataTypeTok{x=}\KeywordTok{expression}\NormalTok{(}\KeywordTok{paste}\NormalTok{(}\StringTok{""}\NormalTok{)))}\OperatorTok{+}
\StringTok{  }\KeywordTok{scale_color_manual}\NormalTok{(}\DataTypeTok{labels =} \KeywordTok{expression}\NormalTok{(}\KeywordTok{paste}\NormalTok{((NH[}\DecValTok{3}\NormalTok{])}\OperatorTok{^}\DecValTok{4}\NormalTok{), }\KeywordTok{paste}\NormalTok{((NO[}\DecValTok{2}\NormalTok{])}\OperatorTok{^}\DecValTok{3}\NormalTok{),}\StringTok{"PO"}\OperatorTok{^}\DecValTok{4}\NormalTok{,}\StringTok{"TN"}\NormalTok{,}\StringTok{"TP"}\NormalTok{),}
                     \DataTypeTok{values =} \KeywordTok{c}\NormalTok{(}\StringTok{"#7fcdbb"}\NormalTok{, }\StringTok{"#41b6c4"}\NormalTok{, }\StringTok{"#1d91c0"}\NormalTok{, }\StringTok{"#225ea8"}\NormalTok{, }\StringTok{"#0c2c84"}\NormalTok{))}\OperatorTok{+}
\CommentTok{# place your additional edits here}
\StringTok{  }\KeywordTok{scale_y_continuous}\NormalTok{(}\DataTypeTok{expand =} \KeywordTok{c}\NormalTok{(}\DecValTok{0}\NormalTok{, }\FloatTok{0.5}\NormalTok{)) }\OperatorTok{+}\StringTok{ }\CommentTok{# expand=c(0,0) has some space above the bars;0=lower space;0.5=upper space}
\StringTok{  }\CommentTok{#scale_color_brewer(palette = "YlGnBu") +}
\StringTok{  }\CommentTok{#scale_color_manual(values = c("#7fcdbb", "#41b6c4", "#1d91c0", "#225ea8", "#0c2c84")) +}
\StringTok{  }\CommentTok{#scale_color_viridis(discrete = TRUE, end = 0.8) +}
\StringTok{  }\KeywordTok{theme}\NormalTok{(}\DataTypeTok{legend.position =} \StringTok{"right"}\NormalTok{)}
\KeywordTok{print}\NormalTok{(Nutrientplot)}
\end{Highlighting}
\end{Shaded}

\includegraphics{09_DataVisualization_files/figure-latex/unnamed-chunk-5-7.pdf}

\hypertarget{multiple-plots-on-a-page}{%
\subsubsection{Multiple plots on a
page}\label{multiple-plots-on-a-page}}

In situations where facets don't fill our needs to place multiple plots
on a page, we can use the package \texttt{cowplot} to arrange plots. The
\texttt{plot\_grid} function is extremely flexible in its ability to
arrange plots in specific configurations. A useful guide can be found
here:
\url{https://cran.r-project.org/web/packages/cowplot/vignettes/introduction.html}.

A useful guide for aligning plots by axis can be found here:
\url{https://wilkelab.org/cowplot/articles/aligning_plots.html}

\begin{Shaded}
\begin{Highlighting}[]
\CommentTok{#install.packages("cowplot")}
\KeywordTok{library}\NormalTok{(cowplot)}
\end{Highlighting}
\end{Shaded}

\begin{verbatim}
## 
## ********************************************************
\end{verbatim}

\begin{verbatim}
## Note: As of version 1.0.0, cowplot does not change the
\end{verbatim}

\begin{verbatim}
##   default ggplot2 theme anymore. To recover the previous
\end{verbatim}

\begin{verbatim}
##   behavior, execute:
##   theme_set(theme_cowplot())
\end{verbatim}

\begin{verbatim}
## ********************************************************
\end{verbatim}

\begin{Shaded}
\begin{Highlighting}[]
\KeywordTok{plot_grid}\NormalTok{(NvsP2, Nutrientplot, }\DataTypeTok{nrow =} \DecValTok{2}\NormalTok{, }\DataTypeTok{align =} \StringTok{'h'}\NormalTok{, }\DataTypeTok{rel_heights =} \KeywordTok{c}\NormalTok{(}\FloatTok{1.25}\NormalTok{, }\DecValTok{1}\NormalTok{))}
\end{Highlighting}
\end{Shaded}

\begin{verbatim}
## Warning: Removed 21587 rows containing missing values (geom_point).
\end{verbatim}

\includegraphics{09_DataVisualization_files/figure-latex/unnamed-chunk-6-1.pdf}

\hypertarget{saving-plots}{%
\subsubsection{Saving plots}\label{saving-plots}}

The \texttt{ggsave} function allows you to save plots in jpg, png, eps,
pdf, tiff, and other formats. The following information can be supplied:

\begin{itemize}
\tightlist
\item
  filename and relative path, with file extension and in quotes
  (required)
\item
  plot object (required)
\item
  width, height, units
\item
  resolution (dpi)
\end{itemize}

For example:
\texttt{ggsave("./Output/PMplot.jpg",\ PMplot.faceted,\ height\ =\ 4,\ width\ =\ 6,\ units\ =\ "in",\ dpi\ =\ 300)}

\hypertarget{visualization-challenge}{%
\subsection{Visualization challenge}\label{visualization-challenge}}

The following graph displays the counts of specific endpoints measured
in neonicotinoid ecotoxicology studies. The way it is visualized,
however, is not effective. Make the following coding changes to improve
the graph:

\begin{enumerate}
\def\labelenumi{\arabic{enumi}.}
\tightlist
\item
  Change the ordering of the ``Endpoint'' factor (function:
  \texttt{reorder}) so that the highest counts are listed first (hint:
  FUN = length)
\item
  Plot the barplot with the reordered factor levels. Add this line of
  code to make the bars show up left to right: scale\_x\_discrete(limits
  = rev(levels(Neonics\$Endpoint)))
\item
  Adjust the x axis labels so they appear at a 45 degree angle.
\item
  Change the color and/or border on the bars. Should you have a
  consistent color across all bars, or a different color for each bar?
\end{enumerate}

\begin{Shaded}
\begin{Highlighting}[]
\NormalTok{Neonics <-}\StringTok{ }\KeywordTok{read.csv}\NormalTok{(}\StringTok{"./Data/Raw/ECOTOX_Neonicotinoids_Insects_raw.csv"}\NormalTok{)}
\KeywordTok{ggplot}\NormalTok{(Neonics) }\OperatorTok{+}
\StringTok{  }\KeywordTok{geom_bar}\NormalTok{(}\KeywordTok{aes}\NormalTok{(}\DataTypeTok{x =}\NormalTok{ Endpoint))}
\end{Highlighting}
\end{Shaded}

\includegraphics{09_DataVisualization_files/figure-latex/unnamed-chunk-7-1.pdf}

\begin{Shaded}
\begin{Highlighting}[]
\CommentTok{#combine plots}
\CommentTok{#install.packages("ggpubr")}
\KeywordTok{library}\NormalTok{(ggpubr)}
\end{Highlighting}
\end{Shaded}

\begin{verbatim}
## Loading required package: magrittr
\end{verbatim}

\begin{verbatim}
## 
## Attaching package: 'magrittr'
\end{verbatim}

\begin{verbatim}
## The following object is masked from 'package:purrr':
## 
##     set_names
\end{verbatim}

\begin{verbatim}
## The following object is masked from 'package:tidyr':
## 
##     extract
\end{verbatim}

\begin{verbatim}
## 
## Attaching package: 'ggpubr'
\end{verbatim}

\begin{verbatim}
## The following object is masked from 'package:cowplot':
## 
##     get_legend
\end{verbatim}

\begin{Shaded}
\begin{Highlighting}[]
\CommentTok{#or}
\KeywordTok{library}\NormalTok{(}\StringTok{"cowplot"}\NormalTok{)}
\end{Highlighting}
\end{Shaded}


\end{document}
